\documentclass[review]{elsarticle}

\usepackage{lineno,hyperref}
\modulolinenumbers[5]

\journal{CIRP Journal of Manufacturing Science and Technology}

%%%%%%%%%%%%%%%%%%%%%%%
%% Elsevier bibliography styles
%%%%%%%%%%%%%%%%%%%%%%%
%% To change the style, put a % in front of the second line of the current style and
%% remove the % from the second line of the style you would like to use.
%%%%%%%%%%%%%%%%%%%%%%%

%% Numbered
%\bibliographystyle{model1-num-names}

%% Numbered without titles
%\bibliographystyle{model1a-num-names}

%% Harvard
%\bibliographystyle{model2-names.bst}\biboptions{authoryear}

%% Vancouver numbered
%\usepackage{numcompress}\bibliographystyle{model3-num-names}

%% Vancouver name/year
%\usepackage{numcompress}\bibliographystyle{model4-names}\biboptions{authoryear}

%% APA style
%\bibliographystyle{model5-names}\biboptions{authoryear}

%% AMA style
%\usepackage{numcompress}\bibliographystyle{model6-num-names}

%% `Elsevier LaTeX' style
\bibliographystyle{elsarticle-num}
%%%%%%%%%%%%%%%%%%%%%%%

\begin{document}

\begin{frontmatter}

\title{An Intuitive Spatial AR System for Augmenting Planar Scenes in Industrial Settings} %\title{Elsevier \LaTeX\ template\tnoteref{mytitlenote}}
%\tnotetext[mytitlenote]{Fully documented templates are available in the elsarticle package on \href{http://www.ctan.org/tex-archive/macros/latex/contrib/elsarticle}{CTAN}.}

%% Group authors per affiliation:
\author{Michael Horn\'{a}\v{c}ek\corref{mycorrespondingauthor}}
\ead{michael.hornacek@tuwien.ac.at}
\address{Human Centered Cyber Physical Production and Assembly Systems, Institute for Management Sciences, TU Vienna, Austria}

\begin{abstract}
Augmented reality (AR) promises to enable use cases in industrial settings that include the embedding of work steps directly into the scene, potentially reducing or altogether obviating the need for workers to refer to instructions in paper form or on a screen. In turn, spatial AR is a form of AR whereby the augmentation of the scene is carried out using a projector, with the advantage or rending the augmentation visible to all onlookers simultaneously without calling for each to wear AR glasses. However, care must be taken to distort the images to be projected in a manner that they appear undistorted to the viewer, since the geometry of the scene as it relates to the geometry of the projector plays a role in how the pixels of the projector’s image plane map to points in the scene. For planar scene geometry (such as a floor, wall, or table), this can be done in a cumbersome manual process called keystone correction, often using software bundled with the projector. We propose a system that produces the same effect by placing the desired augmentations intuitively in accordance with an image of the scene acquired by a camera. We achieve this by distorting the image to be projected using a plane-induced homography computed to produce the effect of projecting the image not from the actual projector pose, but in accordance with a \textit{virtual} projector pose (i) facing directly downwards to the scene plane and (ii) rotated to place the image plane of the virtual projector in line with that of the camera.  
\end{abstract}

\begin{keyword}
Spatial augmented reality (SAR) \sep Industry 4.0
\end{keyword}

\end{frontmatter}

\linenumbers

\section{Introduction}



A central challenge of spatial AR, however,  the geometry of the scene as it relates to the geometry of the projector plays a role in how the pixels of the projector’s image plane map to points in the scene. This is why, unless special care is taken, 

While pointing a projector at a scene and projecting an image is on its own not a challenge, doing so in a manner such that the projected content appear to be naturally part of the scene is more tricky. The intuition for the general problem we face can be drawn from using a projector to project an image to a flat wall: unless the projector faces the wall frontally, the bounds of a projected rectangular image will not appear rectangular, but more generally as a trapezoid (i.e., the image will appear distorted). The cause for this effect is that the geometry of the scene as it relates to the geometry of the projector plays a role in how the pixels of the projector’s image plane map to points in the scene.

The central objective we set for ourselves is to enable correcting for distortion caused by projecting to arbitrary scene geometry in an automatic fashion. This is to be done in a manner that lends itself to practical application in a real-world setting where the scene to be augmented—e.g., an object being gradually assembled and its placement relative to the projector—does not remain static. Finally, the solution is ideally to rely on the existing hardware setup—i.e., projector with steerable mirror system and stereo camera—mounted in the Pilotfabrik  of TU Vienna.

\paragraph{Usage} Once the package is properly installed, you can use the document class \emph{elsarticle} to create a manuscript. Please make sure that your manuscript follows the guidelines in the Guide for Authors of the relevant journal. It is not necessary to typeset your manuscript in exactly the same way as an article, unless you are submitting to a camera-ready copy (CRC) journal.

\paragraph{Functionality} The Elsevier article class is based on the standard article class and supports almost all of the functionality of that class. In addition, it features commands and options to format the
\begin{itemize}
\item document style
\item baselineskip
\item front matter
\item keywords and MSC codes
\item theorems, definitions and proofs
\item lables of enumerations
\item citation style and labeling.
\end{itemize}

\section{Front matter}

The author names and affiliations could be formatted in two ways:
\begin{enumerate}[(1)]
\item Group the authors per affiliation.
\item Use footnotes to indicate the affiliations.
\end{enumerate}
See the front matter of this document for examples. You are recommended to conform your choice to the journal you are submitting to.

\section{Bibliography styles}

There are various bibliography styles available. You can select the style of your choice in the preamble of this document. These styles are Elsevier styles based on standard styles like Harvard and Vancouver. Please use Bib\TeX\ to generate your bibliography and include DOIs whenever available.

Here are two sample references: \cite{Feynman1963118,Dirac1953888}.

\section*{References}

\bibliography{mybibfile}

\end{document}